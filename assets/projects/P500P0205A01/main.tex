\documentclass{book}

\usepackage{hyperref}

\begin{document}
\frontmatter

\title{Emotions and Thought}
\author{Raja W.}
\date{January 2023}
\maketitle

\mainmatter
\chapter{Motivations}

I realized, as I was getting out of a relationship, that I was depressed. The
thought of my romanticism and the thought of entering another relationship
generated deep feelings of depression. I found that many of the thoughts that I
encountered about myself would generate anxiety, depression, and often fear.

Initially, I did not understand the connection between the thoughts and the
emotions that came thereof. I would catch myself laying in my bed, swimming
around in my thoughts, reveling in my self-loathing. I would only get out of bed
to finish work or to eat. It was an effort to go on. In this depression I caught
myself acting irrationally, expressing my pain with a newfound harshness,
lashing out at my friends.

There was one time in my classes that particularly affected me. I was talking to
a girl when the conversation shifted to myself. Many of the things said have
left my mind, however one rested firmly for the rest of the winter: she asked,
"do you talk to girls?" This moment, from her look of pity mixed with
embarrassment of asking the question, to the heads turning in front of her
interested in my answer, felt like years. 

In that moment, I answered: "I feel mildly threatened by that
question."

I spent the holiday break thereafter rather slowly. I would be in my home for
days at a time, often falling into sleep after thinking for hours. Time and time
again, my mind would be brought back to that moment, and I would search for
things to answer instead with. I created a therapist in my mind from the girl
who's words hurt me.

I continued to invent new answers upon new answers, trying to reason with her
and find something that I was happy with. Something that appealed with logic
instead of any emotional self-deprecation. Eventually, I realized that my
pensiveness towards my own love life was due to me believing that I was simply
not fit to be a good boyfriend. I believed that if that were to be the case, I
would end the relationship by me hurting my partner emotionally, and as an
effect, I feel hurt too.

\end{document}


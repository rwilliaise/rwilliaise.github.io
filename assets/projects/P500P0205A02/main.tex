\documentclass[11pt]{memoir}

\begin{document}
\frontmatter

\title{Thought and Emotion: Idea One}
\author{Raja W.}
\date{January 2024}
\maketitle

\mainmatter
\chapter{Preface}
This book is not meant to be a psychological in-depth view of the function of
thought and emotion. This book is simply a compilation of my thoughts regarding
the processes of thought and emotion. I have classified this book as "Idea One,"
as my brain is still young, and my thoughts here will change over time. When
motivated again to write, I will rewrite everything here.

Throughout this book I use terminology in a specific way. When I refer to
thought, I refer implicitly to both conscious and subconscious thought. When I
refer to emotion, I refer to the physiological process of emotion that one feels.

Feeling, in the English language, is nebulous and not well-defined. Often, one
can hear it used to describe one's thoughts: for example, "I feel that nothing
matters at all." This statement does not describe any emotion, even though the
speaker says they "feel" this sentiment.

When a speaker chooses to use "feel" to summarize one's thoughts, it is usually
because the sentiment provided in the statement generates emotions, whether it
be by the thought itself, or by thoughts that are associated with it. In the
statement before, it is possible that the speaker, when thinking about how
nothing matters at all, feels sad. The speaker is trying to both describe
emotion, sadness, and describe why they feel this emotion in one sentence.

\medskip
\begin{flushright}
    \textsc{Raja Williams}
\end{flushright}

\chapter{The Emotional Thinking Machine}
Emotion is an interesting beast. Often, it is difficult to identify the emotions
one is feeling at any one point, which signals to a property of emotion that is
integral to understanding the interaction between emotion and thought. Emotion
is a separate process to thought, however both interact closely enough for this
to be confusing.

Thoughts are generated from stimuli. These thoughts are generated from the
pattern-seeking portion of our brain, which associate different stimuli with
past thoughts. For example, when one sees a dog, subconsciously an association
is reached that the dog is a dog. These stimuli include the senses, which also
include emotion. As well, the stimuli that can generate thought includes thought
itself. Certain thoughts can be associated with other thoughts. 

Thought can create emotion -- however, it cannot be said that emotion directly
creates thoughts. When the mind interprets the emotion the body is feeling, the
mind can make associations to the emotion to different thoughts.

It is often that one comes to the realization of a thought that summarizes both
how he feels and why he feels that. A statement such as "I am worthless"
demonstrates this double-meaning. A listener can empathetically realize the
emotion behind the statement, which could be emotions such as depression or
anger, and realize a general reason why these emotions plague the speaker.

However, the speaker was not thinking the statement for the period of time that
preceded it. In other words, it is not the statement itself that created the
emotions that fueled the speaker to say the statement. The statement itself is a
summarization of the speakers thoughts, and is comprised of the combination of
multiple associations the speaker makes with his self, the object in the
statement. The collection of associations behind the statement are pervasive
patterns of thought that continually incited emotions that the speaker attempts
to portray in the statement.

These patterns of thought can be something like one's self-image. The speaker
may have a low self-esteem, and because of it, feel depressed. The statement "I
am worthless" combines and attempts to portray both the pattern of thought
behind of low self-esteem and the emotions from thoughts that low self-esteem
generates.

This is the integral difference between \textit{feeling} and \textit{belief} in
most contexts. Usually, when a speaker describes feeling, it is the
double-meaning portraying the patterns of thought and the consequences of it
behind the statement. When a speaker describes a belief, it does not necessarily
come with the baggage of the double-meaning. The double-meaning is a side-effect
of equating thoughts as emotions.

\chapter{Emotional Management}
Like thoughts, emotions are generated from stimuli. These stimuli include but
are not limited to thought itself. To manage emotions, it is best to observe the
stimuli that generate the emotion.


\end{document}


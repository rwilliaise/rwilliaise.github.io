\documentclass[10pt]{letter}

\signature{Raja U. Williams \\ 11\textsuperscript{th} grade \\ Lincoln High School}
\address
{
    Lincoln High School\\
    4400 Interlake Ave. N\\
    Seattle, WA 98103
}

\begin{document}
\begin{letter}
{
    The Honorable Patty Murray\\
    154 Russell Senate Office Building\\
    Washington DC 20510
}

\opening{Dear Senator Murray:}

I am a constituent of Congressional District 7, and I have resided here in
Seattle all my life. Although I am not registered to vote, I find it imperative
write to you today not necessarily with a problem of local scale -- rather, an
unease I found with the United States, and the Western paradigms that the U.S
inhabits, and arguably, controls.

The modern industrial society of the United States is a marvel. The
indefatigable drive for continued growth has spurred a globalized society with
the United States being a powerhouse of the modern world and economy. This
letter is not about our status being challenged by China, nor is it of my gripes
of the economy built as a product of the post-World War II society. This letter,
rather, is about tuberculosis, a curable disease. 

Before I begin, let me preface the rest of this letter with one fact: globally,
around every twenty seconds, someone dies of tuberculosis. Keep in mind that
this death is preventable and the disease curable.

Due to the United States' status as an authority of the modern global economy,
there are certain aspects of global society that create dependency upon the
U.S. One of these dependencies is the export of medical instruments and
biotechnology to countries lacking the wherewithal to produce such items
themselves. Private firms export these commodities, usually critical components
of treatment, at unregulated prices to low- and middle-income countries.

An example of this phenomenon was revealed rather recently. Danaher, and its
subsidiary Cepheid, sell molecular diagnostics technology to test for
tuberculosis. The test cartridges that their proprietary machine requires was
\$9.98 per cartridge for low-income countries. This price is outrageous for
those in low-income countries, and many found that they could not seize access
to a test. These tests are far more accurate than examining sputum samples,
which allow for more accurate and faster treatment methodologies, and which is
potentially life-saving.

A pressure campaign initiated and accomplished by several healthcare
organizations eventually led Danaher and Cepheid to lower the price of these
test cartridges by 20\%, to a reduced price of \$7.97. This reduced price,
according to The Global Fund to Fight AIDS, Tuberculosis and Malaria, will allow
more than 5 million more tests to be purchased worldwide each year.

However, tuberculosis activism cannot independently eliminate tuberculosis. In
fact, tuberculosis activism is rather sparse, as general audiences are still
misinformed. Christopher J. Colvin found in his studies that tuberculosis lacks
a "clear set of villains." The  \textit{fons et origo} of many tuberculosis
infections are nebulous problems such as poor housing, poor nutrition, and poor
working conditions. These drivers for infection are alien to United States
audiences.

However, multi-drug resistant tuberculosis continues to even severely affect
those in the United States. MDR-TB and the continued development of it presents
a global health risk of large proportions. As the United States continue to
develop more biotechnology, this phenomenon of price gouging medical exports
will continue without intervention. I posit that the continued health of greater
society and the elimination of tuberculosis as a threat is dependent on the
future regulation on exports of this biotechnology.

\closing{Sincerely,}

\end{letter}
\end{document}

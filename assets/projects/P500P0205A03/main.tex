\documentclass{article}

\usepackage{float}
\usepackage{tikz}

\begin{document}
\title{On Ego}
\date{January 2024}
\author{Raja W.}
\maketitle

\section{Introduction}
The consciousness is a massive, complex and nuanced entity that still has many
of the greatest minds unable to ascertain it's existence. The experience of
consciousness is universal -- thus, defining and describing it is like
describing the sight of a color to the blind, or describing the sound of music
to the deaf. I believe it to be simply not possible, although it is not proven
impossible. In the self-examination of the consciousness we must create an
object representing the consciousness to ascertain and identify patterns of
behavior.

I'm going to use nerd speak to represent my thoughts, because I believe that my
ideas are best understood under the pretense of sets of beliefs or truths. Let
us say that $E$ is the set of all beliefs that this object represents. Let $T$,
then, be the set of all genuine truths about the consciousness.

Is $E$ a subset of $T$? Or in other words, are the beliefs that we hold of our
own consciousness always true? Obviously, no, Often, they are only true to some
extent.

This is easily provable: you could just start believing that you see the color
red incorrectly in some fashion. This belief is not true (as there is no correct
way to see the color red) and thus not in the set of genuine truths $T$, which
implies that $E$ is not always a subset of $T$.

I would argue that a large portion of anxiety, depression, jealousy, envy,
pride, self-loathing, hopelessness, unneeded hopefulness, self-doubt,
self-non-doubt and self-redoubt is primarily based upon these beliefs of your 
self that we hold to be true.

In other words, a large portion of suffering is incurred upon yourself because
you harbor this $E$ set of beliefs. I'll explain why, don't worry.

\section{Subsets of $E$}
It is best to think of $E$ hierarchically, as often it is composed of beliefs
that are, themselves, composed of multiple different beliefs. In other words,
$E$ really is just a bunch of true or false statements about yourself. When you
think about a greater thought about yourself, it is comprised of many different
true or false statements. That is why most beliefs in $E$ only hold some truth,
as only some of the beliefs underlying the thought are actually true.

A diagram of these subsets and the hierarchy that I set up can be found in
Equation \ref{eq:subsets}. $A \to B$ means that the subset of $E$, $A$, has
beliefs which are comprised of, or further broken down into, beliefs in B.

\begin{equation}
E \to D \to U \to S \to ET \to B
\label{eq:subsets}
\end{equation}

The first subset of $E$ is affectionately named $ET$, after the alien with a
suspiciously red tipped finger. No, I lied. It's just beliefs of objective
truths in $E$. Things such as the color of your skin, your current weight, etc.
These are difficult to get wrong. 

However, is $ET$ a subset of $T$? No, again. It is still possible to hold
incorrect beliefs about objective truth. Cite: flat-Earthers. A proof for $ET
\not \subseteq T$ is left as an exercise to the reader. \\

\noindent The second subset is more subtle, and can be broken down often into
beliefs from $ET$. The second subset, called $S$, is beliefs that seem
objectively true as to the statistics relating to those beliefs in $ET$. These
are things like "I am taller than average," "I am heavier than average," or "I
am thinner than average."

"Statistics," at least in the way I use it, does not necessarily imply that you
compare raw numbers to form these beliefs. For example, if you are shorter than
most of the people around you, you come to believe that you are shorter than the
average. It is a fair assumption to make -- but it is not necessarily
statistically (in the mathematical sense) accurate.

You know how it's bad to compare yourself to others? It often only leads to
misery. Now imagine guessing what everyone else (or even smaller groups of
people) are like, and then averaging that out and then comparing yourself to
that. That is $S$.

Is $S$ a subset of $T$? No, as it is based upon statistics, and you can hold
incorrect beliefs about statistics, which implies that it is not an objective
truth and not a subset of $T$. Hell, 90\% of statistics are made up
anyway\footnote{Including, but not limited to, this statistic.}. You get the
idea. The rest of the subsets are provably not always subsets of $T$.

The third subset is $U$, where we ascribe undesirable or desirable traits that
we believe about ourselves to a good or bad, a beautiful or ugly, a true or
false, etc. Obviously, each element is comprised of elements out of $ET$ and
$S$. Examples of elements out of $U$ include: "It is bad I am short," "it is bad
I am lazy," etc.

The fourth subset is $D$, which has desires we believe ourselves to have as a
result of those good or bad in $U$. For example, "I wish I wasn't so short," "I
wish I was thinner," etc. $D$ also contains desires that we identify through
$ET$. For an example, "my stomach rumbles" $\to$ "I'm hungry."

You may notice that the examples I provided in $ET$ are not necessarily of the
consciousness itself, but instead the body. The subset $B$ of $E$ is comprised
of the belief that you are the body. This is the furthest that any belief can be
broken down (to my guess).

I mean it makes sense... you are the body. You are the consciousness who
inhabits your body. Your body and you are one and the same. But it is a belief,
like any other, although it is true. $B$ is the only exception to how the
subsets of $E$ I designated are not subsets of $T$, as it is a purely semantical
truth. You are the body in the same way that you are your consciousness, it is
only a definition of the word.

I include $B$ more as a final destination for the more neurotic: often, your
problems with yourself are not how you act, but instead the properties of your
body itself. You being the body is a belief, which your greater beliefs in $D$
can be built off.

\section{Flattening $E$}

Okay, cool, nerd man, I hear you say (or think\footnote{I do not actually
hear you think. Unless you want me to.} -- whatever). What point do these
subsets have with anything?

That is a good question. These subsets allow the modelling of the breakdown of
beliefs into specific stages, allowing for a simpler way to breakdown and
flatten $E$ into the true or false statements that comprises it.

For example, let us begin with the statement "Darn, I wish I was cooler." This
is obviously an belief in $D$, or $\in D$, so most probably, it is made up of
something $\in U$. Let's say that the something $\in U$ is "It is bad I
stutter." Okay, so this is probably made up of something $\in S$, which let's
say is "I stutter a lot." The "a lot" still implies a statistical idea behind
it, as "a lot" is a comparison to an average perceived by the person. Probably
made up of something $\in ET$, which let's say is "I stuttered in front of my
crush," "I stuttered in front of my friends," etc.

Flattening $E$ allows you to pick apart and eventually obsolete $E$ itself. We
use $E$ as an approximation of $T$, which we use to try and imitate how the
consciousness acts. In the case of $D$, we are trying to imitate the desires of
ourself... but we already know the desires of ourself.

You are already constantly feeling desires, whether it be being hungry, arousal,
thirsty, aroused and thirsty, etc. You don't need to extrapolate desires from
beliefs about yourself, because you already know the desires as an innate state
of being, as that is your consciousness. In other words, you are using $D$ to
imitate yourself.

In the example above, the desire identified was "Darn, I wish I was cooler."
This person that harbors this desire could feel many different emotions from
this statement, but usually depression in the form of a self-loathing. It is a
defensive mechanism -- it is a signal to your body and from your body that
something is wrong (in the form of not being cool enough).

Why does that imply imitation of the self? \\

\noindent Think about a dog. Okay, are you happy? Now think about what she does
(dog owners beware). She functions purely off of a pleasure principle. When she
is hungry, she eats food, and then feels pleasure because she ate food while she
was hungry. Yum! The pleasure builds up the pathways for her to acquire that food
again. This is the principle that training a dog functions under.

She is not imitating herself because she is acting as an agent of her desires
directly. She is just following the instinctual and habitual methodology as to
which she can get pleasure.

You function on this same pleasure principle. You think of a plan to get food
because food makes you feel good when you're hungry, and you will feel better
when you are not hungry anymore. It is also relative -- You pull your hand back
from a hot stove because feeling less pain feels better than before.

Now think about $\in D$. When you think about something $\in D$, you will feel
bad. Based on the pleasure principle, you will start formulating a plan or
follow habit, or some other method to feel better. The problem is, $D$ is not a
subset of $T$, which means that this desire could be entirely unrealistic and/or
unfulfillable, such as the case with "I wish I was cooler." Stopping stuttering
is not necessarily feasible.

That is why $E$ causes suffering, because $\in D$ causes these irreparable
desires that cannot ever be quenched, in a similar fashion to
Tantalus\footnote{Also demonstrated by SCP-198 if you're into that type of
thing.}.

\section{Why flatten?}

This section is the least scientific, and is fully attempting to persuade you to
begin trying to flatten $E$. For one, if you flatten $E$, you will instantly
turn super buff and instantly become super awesome. \\

\noindent No, I lied. The idea behind flattening and dissolving $E$ is that your
actual $T$ doesn't change.

What point is there to flattening $E$ if you already feel awesome? There is two
reasons for this. The thing is, one indisputable fact of $T$ is that it is
always ever-changing. Thus, $E$ is always trying to catch up.

\begin{enumerate}
\item For one, you are just wasting brain power trying to update $E$ to
    constantly catch up with $T$. Get it out of there!

\item For two, there is no guarantee that $\in D$ will consistently stay at a
    point where you feel awesome. An argument of instability, in essence. You
    won't feel awesome forever.
\end{enumerate}

When you flatten $E$, you can look at the beliefs of yourself objectively ($\in
ET$, $\in B$, sometimes higher), and discard any false statements. When the
false statements are discarded, you eventually approach an $E$ which is
actually a subset of $T$. In the moment that $E$ is a subset of $T$, $E$ ceases
to exist as it no longer serves any purpose -- it is repeating information you
already know as a part of being $T$.

In a sense, the issue instead between $E$ and $T$ is instead the elements that
are not $\in T$ while being $\in E$. We know these to be false, as the elements
not $\in T$ are implied to not be genuine truth. Finding these false statements
is the point of flattening $E$.

If $T$ is all truths about your consciousness, does that not mean that $E$ is
contained within $T$? Yes, it is. But that does not imply that $E$ is a subset
of $T$. So the fact that $E$ is in $T$ is pretty much irrelevant.

If you are particularly well read (particularly in the title of this essay), you
might notice that $E$ could be an allegory for "ego." The issue with calling $E$
ego is that 1. it looks ugly in equations, and 2. it implies that I am referring
to the Freudian structural theory when I refer to ego. Which is false: I am not
referring to structural theory.

Obviously, the modern definition of the ego is a lot different than what was
presented in Freud's structural theory. However, I still needed a separate,
clearer entity as to base my interpretations off of. So, I use $E$.

\end{document}

